% Orientações de Programação
%
% Orientações de programação do Sistema de Simulação Numérica e Análises --
% SSNA. O objetivo deste documento é relatar estruturas, detalhes específicos
% e técnicas de programação utilizadas para o desenvolvimento do sistema. Não
% intenciona-se que 100% de tudo seja relatado e sim o que for conveniente
% por questões de padronização, funcionalidade e documentação.
%
% arquivo           : orientacao_de_programacao.tex
% autor             : Marcelo P. Trevizan
% tipo de documento : livre, aberto
% Última alteração	: 25-08-2007
% Versão			: 0.0.0

% ------------------------------------------------------------------------------
% Classe do documento.

\documentclass[a4paper,12pt]{article}

% ------------------------------------------------------------------------------
% Textos pré-definidos.

% Nomes de variáveis de ambiente.
\newcommand{\SSNADIR}{\cod{\$SSNA\_DIR}\xspace}
\newcommand{\SSNATCLLIBDIR}{\cod{\$SSNA\_TCLLIB\_DIR}\xspace}

% ------------------------------------------------------------------------------
% Pacotes.

\usepackage{mestrado}
\onehalfspacing

\title{Orientações de Programação para o SSNA}
\author{}
%\date{}

% ------------------------------------------------------------------------------
% Corpo do documento.

\begin{document}
	\maketitle

    \section{Apresentação}

    Orientações de programação do Sistema de Simulação Numérica e Análises -- SSNA. O objetivo deste documento é relatar estruturas, detalhes específicos e técnicas de programação utilizadas para o desenvolvimento do sistema. Não intenciona-se que 100\% de tudo seja relatado e sim o que for conveniente por questões de padronização, funcionalidade e documentação.

    A seguir, seguem as seções com as orientações para a programação das diversas partes do sistema.

    % ------------------------------------------------------------------------------
    \section{Variáveis de Ambiente}

    O sistema considera as variáveis de ambiente citadas a seguir. Caso não existam, elas serão automaticamente criadas pelo executável \cod{\SSNADIR/ssna}.

    \begin{description}
        \item[\SSNADIR:] Diretório raiz do sistema.
    \end{description}

    \section{Variáveis de Diretorias}

    \ldots

    % ------------------------------------------------------------------------------
    \section{Bibliotecas}

    Atualmente, há bibliotecas em C e em Tcl.

    Para as bibliotecas em Tcl, considerar:

    \begin{itemize}
        \item alocação no diretório \cod{\ldots/lib/tcl};
        \item nomes iniciados por \cod{lib\_};
        \item conter cabeçalho padrão de biblioteca SSNA;
        \item conter trecho de código que evite dupla carga da biblioteca (que pode ocorrer durante a carga de outras bibliotecas ou dos módulos) -- ver \cod{\ldots/lib/tcl/lib\_um\_teste.tcl} para exemplo de código;
        \item as bibliotecas deverão ser carregadas usando-se, preferencialmente, a variável de ambiente \cod{\$SSNA\_TCLLIB} para a localização dos seus correspondentes arquivos.
    \end{itemize}

	% ------------------------------------------------------------------------------
	% Base de dados das bibliografias e formatação bibliográfica.

    %\bibliographystyle{plainnat}
    %\bibliography{$HOME/Trabalhos/Bibliografias/...}

\end{document}

